\section{Data collection methodology}
\label{section:experiments}

We have used the iPhone GPS2IP application to pick up the
GPS coordinates from the internal GPS receiver to get the
current position (latitude and longitude) in GPRMC format.
The application was sending the coordinates to the MacBook
laptop over a UDP socket. The Python script was registering
the coordinates and the timestamps, as well as putting
the data into a file. At the same time, the script was computing
average RTT (in milliseconds) by invoking an external bash script.

Here is the script I was using to do the job:

\begin{Verbatim}[fontsize=\tiny]
from threading import Thread

import socket

from time import sleep
from time import time

UDP_IP = "172.20.10.2"
UDP_PORT = 10000

fd = open("working.log", "w+")

def gps():
	sock = socket.socket(socket.AF_INET, socket.SOCK_DGRAM) 
	sock.bind((UDP_IP, UDP_PORT))
	while True:
		data = sock.recvfrom(1024)
		fd.write(f"{time()} {data[0].decode('ASCII').strip()}\n")
		fd.flush()
		sleep(4)

import subprocess

def icmp():
	while True:
		rtt = subprocess.check_output(["bash", "ping-host.sh"])
		fd.write(f"{time()} {rtt.decode('ASCII').strip()}\n")
		fd.flush()
		sleep(4)

t1 = Thread(target=gps, args=(), daemon=True)
t1.start()

t2 = Thread(target=icmp, args=(), daemon=True)
t2.start()

while True:
	sleep(10)

\end{Verbatim}



%We have collected a long enough trace (1.5 hours long) for the part of the  journey from Tampere to Rovaniemi. We then interpolated 
%the data using the linear interpolator (whenever needed and was possible) and plotted the results on the map. 
%Since the delay between consequtive measurements was 4 seconds we have have interpolated the date using
%the following formula: $t \in \{0.1, 0.2, \ldots 1\}\ s.t.\ x_t = A + (B - A) \cdot t$.
%The source code of the stuff is here~\cite{git}. 


