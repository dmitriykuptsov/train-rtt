\documentclass[conference,10pt,letter]{IEEEtran}

\usepackage{url}
\usepackage{amssymb,amsthm}
\usepackage{graphicx,color}

\usepackage{fancyvrb}

\usepackage{float}

\usepackage{cite}
\usepackage{amsmath}
\usepackage{amssymb}

\usepackage{color, colortbl}
\usepackage{times}
\usepackage{caption}
\usepackage{rotating}
\usepackage{subcaption}

\usepackage{balance}

\newtheorem{theorem}{Theorem}
\newtheorem{example}{Example}
\newtheorem{definition}{Definition}
\newtheorem{lemma}{Lemma}

\newcommand{\XXXnote}[1]{{\bf\color{red} XXX: #1}}
\newcommand{\YYYnote}[1]{{\bf\color{red} YYY: #1}}
\newcommand*{\etal}{{\it et al.}}

\newcommand{\eat}[1]{}
\newcommand{\bi}{\begin{itemize}}
\newcommand{\ei}{\end{itemize}}
\newcommand{\im}{\item}
\newcommand{\eg}{{\it e.g.}\xspace}
\newcommand{\ie}{{\it i.e.}\xspace}
\newcommand{\etc}{{\it etc.}\xspace}
%\newcommand{\em}[1]{\it}

\def\P{\mathop{\mathsf{P}}}
\def\E{\mathop{\mathsf{E}}}

\begin{document}
\sloppy
\title{Measuring the network RTT in a moving train: Fun project}
\maketitle
\begin{abstract}
What to do on the train if the journey takes hours?
Of course, measure the packet latency to the selected server in
the Internet and plotting the data after interpolation on the map.
Fun? Yes, it is.
\end{abstract}

\input intro.tex 
\input methodology.tex
\input results.tex
%\input conclusions.tex

\balance
\bibliographystyle{abbrv}
\bibliography{mybib}

\end{document}
